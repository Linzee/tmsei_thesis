%%%%%%%%%%%%%%%%%%%%%%%%%%%%%%%%%%%%%%%%%%%%%%%%%%%%%%%%%%%%%%%%%%%%
%% I, the copyright holder of this work, release this work into the
%% public domain. This applies worldwide. In some countries this may
%% not be legally possible; if so: I grant anyone the right to use
%% this work for any purpose, without any conditions, unless such
%% conditions are required by law.
%%%%%%%%%%%%%%%%%%%%%%%%%%%%%%%%%%%%%%%%%%%%%%%%%%%%%%%%%%%%%%%%%%%%

% Content:

%  general
%   Similarity
%   Possible usage
%   Metric types
%   (Scicentific) Question, Goal
% data specific
%   Robotanik
%   Evaluation
%   Collecting expert data
% implementation specific
%   my metrics
%   Tool
%   Data Format

\documentclass[
  digital, %% This option enables the default options for the
           %% digital version of a document. Replace with `printed`
           %% to enable the default options for the printed version
           %% of a document.
  table,   %% Causes the coloring of tables. Replace with `notable`
           %% to restore plain tables.
  lof,     %% Prints the List of Figures. Replace with `nolof` to
           %% hide the List of Figures.
  lot,     %% Prints the List of Tables. Replace with `nolot` to
           %% hide the List of Tables.
  %% More options are listed in the user guide at
  %% <http://mirrors.ctan.org/macros/latex/contrib/fithesis/guide/mu/fi.pdf>.
]{fithesis3}
%% The following section sets up the locales used in the thesis.
\usepackage[resetfonts]{cmap} %% We need to load the T2A font encoding
\usepackage[T1,T2A]{fontenc}  %% to use the Cyrillic fonts with Russian texts.
\usepackage[
  main=english, %% By using `czech` or `slovak` as the main locale
                %% instead of `english`, you can typeset the thesis
                %% in either Czech or Slovak, respectively.
  english, german, russian, czech, slovak %% The additional keys allow
]{babel}        %% foreign texts to be typeset as follows:
%%
%%   \begin{otherlanguage}{german}  ... \end{otherlanguage}
%%   \begin{otherlanguage}{russian} ... \end{otherlanguage}
%%   \begin{otherlanguage}{czech}   ... \end{otherlanguage}
%%   \begin{otherlanguage}{slovak}  ... \end{otherlanguage}
%%
%% For non-Latin scripts, it may be necessary to load additional
%% fonts:
\usepackage{paratype}
\def\textrussian#1{{\usefont{T2A}{PTSerif-TLF}{m}{rm}#1}}
%%
%% The following section sets up the metadata of the thesis.
\thesissetup{
    date          = \the\year/\the\month/\the\day,
    university    = mu,
    faculty       = fi,
    type          = bc,
    author        = Dominik Gmiterko,
    gender        = m,
    advisor       = John Smith,
    title         = {Similarity of programming problems},
    TeXtitle      = {Similarity of programming problems},
    keywords      = {similarity, metrics, programming, keyword2, ...},
    TeXkeywords   = {similarity, metrics, programming, keyword2, \ldots},
    abstract      = {This is the abstract of my thesis, which can

                     span multiple paragraphs.},
    thanks        = {These are the acknowledgements for my thesis, which can

                     span multiple paragraphs.},
    bib           = thesis.bib,
}
\usepackage{makeidx}      %% The `makeidx` package contains
\makeindex                %% helper commands for index typesetting.
%% These additional packages are used within the document:
\usepackage{paralist} %% Compact list environments
\usepackage{amsmath}  %% Mathematics
\usepackage{amsthm}
\usepackage{amsfonts}
\usepackage{url}      %% Hyperlinks
\usepackage{markdown} %% Lightweight markup
\usepackage{listings} %% Source code highlighting
\lstset{
  basicstyle      = \ttfamily,%
  identifierstyle = \color{black},%
  keywordstyle    = \color{blue},%
  keywordstyle    = {[2]\color{cyan}},%
  keywordstyle    = {[3]\color{olive}},%
  stringstyle     = \color{teal},%
  commentstyle    = \itshape\color{magenta}}
\usepackage{floatrow} %% Putting captions above tables
\floatsetup[table]{capposition=top}
\begin{document}
\chapter*{Introduction}
\addcontentsline{toc}{chapter}{Introduction}

Define a metric for measuring similarity of programming problems
for introductory programming

\begin{markdown*}{%
  hybrid,
  definitionLists,
  footnotes,
  inlineFootnotes,
  hashEnumerators,
  fencedCode,
  citations,
  citationNbsps,
}

%
% Introduction
%

% TODO co je citatelov ciel, na co sa moze tesit

%
% 1. General, about similarity of programming problems
%

# Similarity

There is a lot of research about similarity in many different fields like bioinformatics (sequence alignment, similarity matrix of proteins), data retrieval (document similarity), plagiarism detection and many more. On other hand there is only a few mentions about comparing programming problems currently published.

recommender systems

% TODO
not problmes but source code similarity -> plagiatorism ~[@beth2014comparison], source quality control ~[@]

% TODO?
> Similarity Techniques for Detecting Source Code Plagiarism
> Modeling How Students Learn to Program
% CITE from similarity techniques:
% The abstract nature of computer source code, however, limits the feasibility of applying such tools to computational artifacts
%...different approaches has to be used
% possible ambiguities in program semantics, other features, such as arbitrary identifier names, variable whitespace, and nonlinear sequencing of code, present difficulties in textual similarity analysis unique to program source code.
% TODO
"leveraging human intelligence” instead of its use for automatic intelligent method ~[@baker2016stupid]

## Programming environments

% TODO
I would like to define what programming environment and programming problems is for me..

When we talk about programming problems we in most cases mean problems created for interactive programming environments for introductory programming. Some examples of environments are [Robotanik](https://tutor.fi.muni.cz/), [Karel](http://stanford.edu/~cpiech/karel/ide.html), Lightbot.

Problems in this environments are specified by grid with tiles which have different meaning. Users are supposed to write program for some entity to fulfil given task. Most common task is to visit all tiles of some type. In Robotanik student are asked to build program guiding robot to collect all flowers.

Some programming environments try to constrain student in order to produce more creative solutions. One possible limitation is allowing student to use only subset of all available commands. Another approach is to limit length of program.

## Possible usage

When we define metrics for measuring similarity of programming problems we can use them for different purposes. Most of them are useful for adaptive programming environments. First possible usage is recommendation of problems for student to solve. We do not want to recommend very similar problems to those that were solved without any problems. However when student struggled system should recommend more of similar problems.

% TODO

When hints ~[@hosseini2017study]

% TODO A study of concept-based similarity approaches for recommending program examples

% TODO This can be more general

Another idea I am quite found of is automatic construction of user interface. When we have large amount of problems without any present categorization we can use problem similarity to construct categories for student to select from.

clusters can serve as a basis for knowledge component  definition  or  refinemen ~[@pelanekmeasuring]

Authors of systems can use our metrics for gaining insight of problem pool. It is possible to detect redundant problems, maybe even tell which problems are missing. One way of achieving this is plotting problems to plane and displaying it to author.

## Metric types

We can define different metrics for comparing programming problems. It is possible to divide metrics into 3 categories based on data that is used.

**Problem statement** in interactive programming environments can be used to collect features used in problem. Similarity is mostly computed using distance of vectors describing some features of problem. This category includes usage of tiles in problem, allowed commands, size of level, and more.

**Solution syntax** metrics differ greatly with method used by users to input solutions. We can divide them to textual which are written in commonly used programming languages like Java and visual.

% TODO
Metrics comparing textual solutions can use all methods known from information retrieval (tf–idf) on source-code text. Another possible approach is to use AST (abstract syntax tree) which is available for almost all of this programming languages. In both cases we can compute similarity as edit distance (Levenshtein edit distance and Tree edit distance respectively), presence of keywords, count of keywords and many more.

%%% There exist many well-defined metrics to measure simi-
%%% larity. Each of these is typically specified in terms of an in-
%%% verse characteristic—edit distance. Edit distance is defined
%%% as the minimum number of applications from a defined set
%%% of operations necessary to transform one instance of a struc-
%%% ture to another. Variations in edit distance correspond to
%%% differences in structures, the set of defined operations, and
%%% relative weights among their applications.

% TODO neprepisane! nekonzistentne s blokom nadtym
Some programming environments does not support text as solution but some simpler alternative is provided for learners. For example Robotanik uses commands (blocks) that can be placed in rows representing functions. Programming enthronements Robomise and Scratch use [Blockly](https://developers.google.com/blockly/) (like) user interface.
It is usually simpler to convert solutions into some string or tree which can be used with standard methods.

All metrics defined for single solution can also be extended to collect information from multiple students and all their attempts.

% TODO one possible implementeation is to use averege of clacualted features

**Performance based** metrics requires having collected data about students while solving problems. Example of data that falls in this category is solution time, number of attempts, hints used. With this type of metrics we compute similarity as correlation of one or multiple statistics.

## (Scicentific) Question, Goal

In the rest of the >>>thesis<<< we will try to answer several questions.

1. Do different (types of) metrics correlate? I.e., is there "one underlying similarity" or do they rather measure different aspects of problem similarity? If there are multiple aspects, what is the relation between them? How do we use them in practical applications?

2. Use "human data" (expert/student judgment) experiment -- collect judgment about (dis)similarity of items and compare with automatic metrics.
paired comparisons: similarity of A and B on scale 1 to 10
triplets: Given problem A, which is more similar: B or C?

3. Tool which will


%
% 2. Data specific
%

# Robotanik

Most of the experiments were executed on data collected from programming environment Robotanik available at [tutor.fi.muni.cz](http://tutor.fi.muni.cz/).

## Evaluation

### Compare metrics

### Collecting crowd-truth


%
% 3. Implementation specific
%

# Tool

## My metrics
### Problem statement

### Solutions

### Performance data

## Tool capabilities

\end{markdown*}

  \makeatletter\thesis@blocks@clear\makeatother
  \phantomsection %% Print the index and insert it into the
  \addcontentsline{toc}{chapter}{\indexname} %% table of contents.
  \printindex

\appendix %% Start the appendices.
\chapter{An appendix}
Here you can insert the appendices of your thesis.

\end{document}
