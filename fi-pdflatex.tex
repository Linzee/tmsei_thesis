%%%%%%%%%%%%%%%%%%%%%%%%%%%%%%%%%%%%%%%%%%%%%%%%%%%%%%%%%%%%%%%%%%%%
%% I, the copyright holder of this work, release this work into the
%% public domain. This applies worldwide. In some countries this may
%% not be legally possible; if so: I grant anyone the right to use
%% this work for any purpose, without any conditions, unless such
%% conditions are required by law.
%%%%%%%%%%%%%%%%%%%%%%%%%%%%%%%%%%%%%%%%%%%%%%%%%%%%%%%%%%%%%%%%%%%%

\documentclass[
  digital, %% This option enables the default options for the
           %% digital version of a document. Replace with `printed`
           %% to enable the default options for the printed version
           %% of a document.
  table,   %% Causes the coloring of tables. Replace with `notable`
           %% to restore plain tables.
  nolof,     %% Prints the List of Figures. Replace with `nolof` to
           %% hide the List of Figures.
  nolot,     %% Prints the List of Tables. Replace with `nolot` to
           %% hide the List of Tables.
  nocover
  %% More options are listed in the user guide at
  %% <http://mirrors.ctan.org/macros/latex/contrib/fithesis/guide/mu/fi.pdf>.
]{fithesis3}
%% The following section sets up the locales used in the thesis.
\usepackage[resetfonts]{cmap} %% We need to load the T2A font encoding
\usepackage[T1,T2A]{fontenc}  %% to use the Cyrillic fonts with Russian texts.
\usepackage[
  main=english, %% By using `czech` or `slovak` as the main locale
                %% instead of `english`, you can typeset the thesis
                %% in either Czech or Slovak, respectively.
  english, german, russian, czech, slovak %% The additional keys allow
]{babel}        %% foreign texts to be typeset as follows:
%%
%%   \begin{otherlanguage}{german}  ... \end{otherlanguage}
%%   \begin{otherlanguage}{russian} ... \end{otherlanguage}
%%   \begin{otherlanguage}{czech}   ... \end{otherlanguage}
%%   \begin{otherlanguage}{slovak}  ... \end{otherlanguage}
%%
%% For non-Latin scripts, it may be necessary to load additional
%% fonts:
\usepackage{paratype}
\def\textrussian#1{{\usefont{T2A}{PTSerif-TLF}{m}{rm}#1}}
%%
%% The following section sets up the metadata of the thesis.
\thesissetup{
    date          = \the\year/\the\month/\the\day,
    university    = mu,
    faculty       = fi,
    type          = bc,
    author        = Dominik Gmiterko,
    gender        = m,
    advisor       = Radek Pelánek,
    title         = {Similarity of programming problems},
    TeXtitle      = {Similarity of programming problems},
    keywords      = {similarity, metrics, programming, keyword2, ...},
    TeXkeywords   = {similarity, metrics, programming, keyword2, \ldots},
    abstract      = {This is the abstract of my thesis, which can

                     span multiple paragraphs.},
    thanks        = {These are the acknowledgements for my thesis, which can

                     span multiple paragraphs.},
    bib           = thesis.bib,
}
\usepackage{makeidx}      %% The `makeidx` package contains
\makeindex                %% helper commands for index typesetting.
%% These additional packages are used within the document:
\usepackage{paralist} %% Compact list environments
\usepackage{amsmath}  %% Mathematics
\usepackage{amsthm}
\usepackage{amsfonts}
\usepackage{url}      %% Hyperlinks
\usepackage{markdown} %% Lightweight markup
\usepackage{listings} %% Source code highlighting
\lstset{
  basicstyle      = \ttfamily,%
  identifierstyle = \color{black},%
  keywordstyle    = \color{blue},%
  keywordstyle    = {[2]\color{cyan}},%
  keywordstyle    = {[3]\color{olive}},%
  stringstyle     = \color{teal},%
  commentstyle    = \itshape\color{magenta}}
\usepackage{floatrow} %% Putting captions above tables
\floatsetup[table]{capposition=top}
\begin{document}

%
% Oficialne ZADANI v.1
%
% Cílem práce je prozkoumat metody pro měření podobnosti výukových položek na
% základě dat o odpovědích studentů. Práce navazuje na předchozí výzkum v rámci
% skupiny Adaptive Learning. Cílem práce je kriticky prozkoumat dříve navržené
% metody, zejména s ohledem na nevyjasněné aspekty jejich chování na reálných
% datech (neočekávané pravidelnosti v distribucích hodnot podobnosti). Na
% základě získaného vhledu budou navrženy upravené metody nebo doporučení pro
% praktický postup.
%
%                                                                   KISS
%


\chapter*{Introduction}
\addcontentsline{toc}{chapter}{Introduction}

\begin{markdown*}{%
  hybrid,
  definitionLists,
  footnotes,
  inlineFootnotes,
  hashEnumerators,
  fencedCode,
  citations,
  citationNbsps,
}

% --------------------------- %
% Introduction                %
% --------------------------- %

% interactive educational systems

Tutoring systems are computer-based systems designed to introduce users into various domains.
They usually have large amount of items which enables them to provide personalized experience. To maintain this large pool of items efficiently we need to be able to decide which items are useful and which are not.

% focus on systems with large amount of items

% one possible method, using similarity of items

% goal of this work

% structure of this work

Besides Introduction and Conclusion chapters, this thesis is structured into three additional chapters. First chapter talks in general about problem of measuring similarity of programming problems. It explains difference between program and programming problem, which data we have available and techniques used for measuring similarity of problems. Second chapter advances level deeper and describe everything what is specific to data we used. First part of chapter describes programming environment of Robotanik and data from it. Second part focuses in detail on metrics we used in experiments. Last chapter gives overview of implementation and usage of metrics and their evaluation.

# Similarity

% --------------------------- %
% Similarity                  %
% --------------------------- %

% Intro to chapter

%% Structure

In this chapter we will talk in general about questions in learning systems, and computing their similarity. Most of the chapter focuses on explaining what kinds of data are available when comparing questions in learning systems and techniques to do so. Last section describes goals of the thesis.

%% Using similarity in related fields

A lot of research has been dedicated to similarity in many different fields computer science like bioinformatics (sequence alignment, similarity matrix of proteins), information retrieval (document similarity), plagiarism detection and many more.

%TODO este sme nepovedali ze pouzivame performance, presunut

One closely related area is recommender systems which differs from problem similarity only slightly. Both areas are distinguishing users and items. Only difference is that we know how well user did while solving specific item and recommender systems use rating of the items.

%%% Main difference from educational systems

Main difference is that we can use more data about problem. We also have some problem statement and data about performance of students when solving problem.

## Items

% Items

%% Why this term

In this work we use the term “items” (problems, questions, assignments) when we refer to single entry in educational system which users can answer to. Since many aspects of this work are generally applicable we decided to use this general term. In some learning systems this can refer to simple choice from two options in another complex tasks which user solves in matter of minutes.
On other side of the spectrum are systems for teaching introductional programming. Users tend to spend few minutes solving each task and there is fewer of them.

%% Data sources

To further specify the context of our research, we will describe characteristics of items. For computing similarity of items it is most important knowing which data are available to us. Therefore we describe items by sources of data can be used for measuring similarity.

- **Item statement:** specification of the item that a learner should solve, e.g., as a natural language description of the task.

- **Item solutions:** details about solutions obtained from learners or sample solution to item.

- **Learner's performance:** for example item solving times, correctness of answer, number of attempts needed.

This description of item is broad enough to cover most of learning systems. In next chapters we will discuss two systems in particular - umimecesky a umimematiku.

## Why is similarity of items useful
% Why is similarity of items useful

As we mentioned previously key part of learning solving of educational items.

## Computing similarity of items
% Computing similarity of items

The general approach to measuring and using similarity of educational items

## Used datasets
% Datasets

% types of data (real, simualted), why?, real, simulated

In our analysis we use both real data from educational system and simulated data. There is a reason why use both as only real-world data are useful for concluding any results. However evaluation of this data is often complicated as we do not know truth about many of their aspects. That's why we used simulated data for validating some of our conclusions. We will talk more in depth about how we generated simulated data in next chapter when describing their specific usage.

Most of used real-world data comes from system [Umíme česky](https://umimecesky.cz/). Later we have validated our results by also using data from its sibling system [Umíme matiku](https://umimematiku.cz/). We think it is useful as data come from another context but are provided in same format and therefore can be used directly in previously created tools.

### Umíme česky
% czech grammar, "fill-in-the-blank" with two choices, correctness and time, multiple grammar concepts, example of one exercise

[Umíme česky](https://umimecesky.cz/) is system for practice of Czech grammar. System contains multiple exercise types, but in our analysis we use only one exercise - simple "fill-in-the-blank" with two possible answers. This type of exercise can be then viewed by student in multiple ways.
%TODO all views of exercise

!["fill-in-the-blank" example question](img/umimecesky_doplnovacka)

We focused only on "fill-in-the-blank" exercises but they can still be used to train many concepts of Czech grammar.
%TODO different concepts in grammar

### Umíme matiku


# Evaulation

% --------------------------- %
% Evaulation                  %
% --------------------------- %

% Description of common projection output

%% why is projection useful

it is hard to say anything about data when tehre is a lot of it
we especialy focused on systems which consist of TODO1000s of solvable items and even more users
in cases like this it is not possible to look at data about each item individually
possible with projection of many-dimensional data into 2 dimensions.

%% similarity and projection

Many different data can be used for computing projections
we want from this projection to have some attributes
the most important one is that items in projection should be closer to each other if they are more similar to each other than if they are not similar at all

%% how does projection look like, possibel answers and levels

Figure \ref{common_projection} shows how common projection looks like. This particular projection consist of 273 items TODOpatriacich into one TODOknowledge component.

Most TODOknowledge components in system Umíme česky are spitted into multiple difficulty levels.

Items in displayed concept are divided into three levels. This is shown in image with three different colors (first level with red, second with green and third in blue).

![Basic projection of one context in Umíme česky\label{common_projection}](img/common_projection)

% Regularities

After looking at image you can notice some regularities.


%% How is projection (similarity) computed

To better understand how was this image created we have to understand how it is computed. We will describe it following section. This is default work-flow we used in most cases. Whenever we don't specify otherwise all projections were produced using this work-flow.

%%% Performance matrix

1. First step we have to do is converting raw logged information about user answers to performance matrix. This matrix consist of entries for each user-item pair. Columns of the matrix are items in educational system and each row of the matrix contains data about single user's performance. In most cases we used correctness of first users answer to specific item as his performance. This means value 1.0 in case of correct answer and 0.0 for incorrect. Another possible choice is to incorporate user solving time into this value. TODO possible choices are described in [] Performance matrix is relatively sparse as it is not common for users to solve all the items in the system.

%%% Similarity matrix

2. Next step is computing similarity matrix. Matrix $S$ is square matrix where each position $S_{ij}$ denotes similarity of items $i$ and $j$. In our case similarity is computed as correlation of two columns $i$ and $j$ in performance matrix. This means we look at users who solved both items $i$ and $j$ and compute Pearson correlation of this user performance entries.

%%% PCA, (TSNE)

3. The last (optional) step is producing 2D projection. This can be achieved by using many techniques for computing low dimensional projections. We choose principal component analysis (PCA). Two principal components are then used for 2D visualizations.

%%% Why this work-flow, performance data (not item statement)

We choose this specific work-flow as we think it is utilizing data about items which hold most information about their similarity. As other possible choices are item statement and solutions provided by students they do not hold as much information. Item statements in our particular exercises consist only of few Czech words. Also student solution is only choice from two provided options. Item statement and solutions can be used more effectively in other contexts like programming, mathematics, physics, or chemistry.

This choice of work-flow is also relatively simple and easy to understand. It consist of few steps which can be studied separately and interchanged.

## Answer regularity
% Answer regularity
%% Description of problem, items are divided into clusters based correct answer (i/y) when there is no information about this in data used for computing similarity

%% Quantification for different concepts

%% Used metric type doesn't affect results

%% Total similarity

## Level regularity
% Level regularity
%% Description of problem, 3 clusters based on 3 levels of questions difficulty, how users tend to solve them

%% Missing data, structure

%%% Simulated data experiment

%% Quantification of level clustering, split and compare with truth

## Another context
% Another context

%% Replicated in another context of math

%% (re)Usage in other concepts

# Conclusion

% --------------------------- %
% Conclusion                  %
% --------------------------- %



\end{markdown*}

  \makeatletter\thesis@blocks@clear\makeatother
  \phantomsection %% Print the index and insert it into the
  \addcontentsline{toc}{chapter}{\indexname} %% table of contents.
  \printindex

\appendix %% Start the appendices.
\chapter{An appendix}
Here you can insert the appendices of your thesis.

\end{document}
