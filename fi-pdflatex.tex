%%%%%%%%%%%%%%%%%%%%%%%%%%%%%%%%%%%%%%%%%%%%%%%%%%%%%%%%%%%%%%%%%%%%
%% I, the copyright holder of this work, release this work into the
%% public domain. This applies worldwide. In some countries this may
%% not be legally possible; if so: I grant anyone the right to use
%% this work for any purpose, without any conditions, unless such
%% conditions are required by law.
%%%%%%%%%%%%%%%%%%%%%%%%%%%%%%%%%%%%%%%%%%%%%%%%%%%%%%%%%%%%%%%%%%%%

\documentclass[
  digital, %% This option enables the default options for the
           %% digital version of a document. Replace with `printed`
           %% to enable the default options for the printed version
           %% of a document.
  table,   %% Causes the coloring of tables. Replace with `notable`
           %% to restore plain tables.
  nolof,     %% Prints the List of Figures. Replace with `nolof` to
           %% hide the List of Figures.
  nolot,     %% Prints the List of Tables. Replace with `nolot` to
           %% hide the List of Tables.
  nocover
  %% More options are listed in the user guide at
  %% <http://mirrors.ctan.org/macros/latex/contrib/fithesis/guide/mu/fi.pdf>.
]{fithesis3}
%% The following section sets up the locales used in the thesis.
\usepackage[resetfonts]{cmap} %% We need to load the T2A font encoding
\usepackage[T1,T2A]{fontenc}  %% to use the Cyrillic fonts with Russian texts.
\usepackage[
  main=english, %% By using `czech` or `slovak` as the main locale
                %% instead of `english`, you can typeset the thesis
                %% in either Czech or Slovak, respectively.
  english, german, russian, czech, slovak %% The additional keys allow
]{babel}        %% foreign texts to be typeset as follows:
%%
%%   \begin{otherlanguage}{german}  ... \end{otherlanguage}
%%   \begin{otherlanguage}{russian} ... \end{otherlanguage}
%%   \begin{otherlanguage}{czech}   ... \end{otherlanguage}
%%   \begin{otherlanguage}{slovak}  ... \end{otherlanguage}
%%
%% For non-Latin scripts, it may be necessary to load additional
%% fonts:
\usepackage{paratype}
\def\textrussian#1{{\usefont{T2A}{PTSerif-TLF}{m}{rm}#1}}
%%
%% The following section sets up the metadata of the thesis.
\thesissetup{
    date          = \the\year/\the\month/\the\day,
    university    = mu,
    faculty       = fi,
    type          = bc,
    author        = Dominik Gmiterko,
    gender        = m,
    advisor       = Radek Pelánek,
    title         = {Similarity of programming problems},
    TeXtitle      = {Similarity of programming problems},
    keywords      = {similarity, metrics, programming, keyword2, ...},
    TeXkeywords   = {similarity, metrics, programming, keyword2, \ldots},
    abstract      = {This is the abstract of my thesis, which can

                     span multiple paragraphs.},
    thanks        = {These are the acknowledgements for my thesis, which can

                     span multiple paragraphs.},
    bib           = thesis.bib,
}
\usepackage{makeidx}      %% The `makeidx` package contains
\makeindex                %% helper commands for index typesetting.
%% These additional packages are used within the document:
\usepackage{paralist} %% Compact list environments
\usepackage{amsmath}  %% Mathematics
\usepackage{amsthm}
\usepackage{amsfonts}
\usepackage{url}      %% Hyperlinks
\usepackage{markdown} %% Lightweight markup
\usepackage{listings} %% Source code highlighting
\lstset{
  basicstyle      = \ttfamily,%
  identifierstyle = \color{black},%
  keywordstyle    = \color{blue},%
  keywordstyle    = {[2]\color{cyan}},%
  keywordstyle    = {[3]\color{olive}},%
  stringstyle     = \color{teal},%
  commentstyle    = \itshape\color{magenta}}
\usepackage{floatrow} %% Putting captions above tables
\floatsetup[table]{capposition=top}
\begin{document}

%
% Oficialne ZADANI v.1
%
% Cílem práce je prozkoumat metody pro měření podobnosti výukových položek na
% základě dat o odpovědích studentů. Práce navazuje na předchozí výzkum v rámci
% skupiny Adaptive Learning. Cílem práce je kriticky prozkoumat dříve navržené
% metody, zejména s ohledem na nevyjasněné aspekty jejich chování na reálných
% datech (neočekávané pravidelnosti v distribucích hodnot podobnosti). Na
% základě získaného vhledu budou navrženy upravené metody nebo doporučení pro
% praktický postup.
%
%                                                                   KISS
%


\chapter*{Introduction}
\addcontentsline{toc}{chapter}{Introduction}

\begin{markdown*}{%
  hybrid,
  definitionLists,
  footnotes,
  inlineFootnotes,
  hashEnumerators,
  fencedCode,
  citations,
  citationNbsps,
}

% --------------------------- %
% Introduction                %
% --------------------------- %

% interactive educational systems
Tutoring systems are computer-based systems designed to introduce users into various domains.
They usually have large amount of items which enables them to provide personalized experience. To maintain this large pool of items efficiently we need to be able to decide which items are useful and which are not.

% focus on systems with large amount of items

% one possible method, using similarity of items

% goal of this work

% structure of this work
Besides Introduction and Conclusion chapters, this thesis is structured into three additional chapters. First chapter talks in general about problem of measuring similarity of programming problems. It explains difference between program and programming problem, which data we have available and techniques used for measuring similarity of problems. Second chapter advances level deeper and describe everything what is specific to data we used. First part of chapter describes programming environment of Robotanik and data from it. Second part focuses in detail on metrics we used in experiments. Last chapter gives overview of implementation and usage of metrics and their evaluation.

# Similarity

% --------------------------- %
% Similarity                  %
% --------------------------- %

% Intro to chapter
%% Structure
In this chapter we will talk in general about questions in learning systems, and computing their similarity. Most of the chapter focuses on explaining what kinds of data are available when comparing questions in learning systems and techniques to do so. Last section describes goals of the thesis.

%% Using similarity in related fields
A lot of research has been dedicated to similarity in many different fields computer science like bioinformatics (sequence alignment, similarity matrix of proteins), information retrieval (document similarity), plagiarism detection and many more.

%TODO este sme nepovedali ze pouzivame performance, presunut
One closely related area is recommender systems which differs from problem similarity only slightly. Both areas are distinguishing users and items. Only difference is that we know how well user did while solving specific item and recommender systems use rating of the items.

%%% Main difference from educational systems
Main difference is that we can use more data about problem. We also have some problem statement and data about performance of students when solving problem.

## Items
% Items
%% Why this term
In this work we use the term “item” when we refer to single entry in educational system which users can answer to. Since many aspects of this work are generally applicable we decided to use this general term. In some learning systems this can refer to simple choice from two options in another complex tasks which user solves in matter of minutes.
On other side of the spectrum are systems for teaching introductional programming. Users tend to spend few minutes solving each task and there is fewer of them.

%% Data sources
To further specify the context of our research, we will describe characteristics of items. For computing similarity of items it is most important knowing which data are available to us. Therefore we describe items by sources of data can be used for measuring similarity.

- **Item statement:** specification of the item that a learner should solve, e.g., as a natural language description of the task.

- **Item solutions:** details about solutions obtained from learners or sample solution to item. 

- **Learner's performance:**

## Why is similarity of items useful
% Why is similarity of items useful


## Computing similarity of items
% Computing similarity of items
The general approach to measuring and using similarity of
educational items

## Used datasets
% Datasets

# Evaulation

% --------------------------- %
% Evaulation                  %
% --------------------------- %


# Conclusion

% --------------------------- %
% Conclusion                  %
% --------------------------- %



\end{markdown*}

  \makeatletter\thesis@blocks@clear\makeatother
  \phantomsection %% Print the index and insert it into the
  \addcontentsline{toc}{chapter}{\indexname} %% table of contents.
  \printindex

\appendix %% Start the appendices.
\chapter{An appendix}
Here you can insert the appendices of your thesis.

\end{document}
